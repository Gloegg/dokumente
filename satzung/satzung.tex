\documentclass[ngerman]{scrartcl}
\usepackage[utf8]{inputenc}
\usepackage[ngerman]{babel}
\usepackage[T1]{fontenc}
\usepackage{microtype}
\usepackage{palatino}
\usepackage{tabularx}
%\usepackage{times}
\begin{document}
\renewcommand*{\othersectionlevelsformat}[3]{\S#3\autodot\enskip}
\begin{titlepage}
\centerline{\Large Satzung}
\begin{center}
{\Huge fNordeingang}\\[5mm]
\vfill
--- beschlossen durch die Mitgliederversammlung am 2013-01-18 ---
\end{center}
\end{titlepage}
\section{Name, Sitz, Geschäftsjahr} \label{sec:name_sitz_geschaeftsjahr}
\begin{enumerate}
 \item Der Verein führt den Namen "`fNordeingang"'. Er hat
 seinen Sitz in Neuss, wird in das Vereinsregister
 eingetragen und trägt dann den Zusatz e. V.
 \item Das Geschäftsjahr des Vereins ist das Kalenderjahr.
\end{enumerate}
\section{Zweck und Gemeinnützigkeit} \label{sec:zweck}
\begin{enumerate}
 \item Die Informationsgesellschaft des 21. Jahrhunderts ist ohne Computer und ihre Vernetzung nicht mehr denkbar. Die Einsatzmöglichkeiten der digitalen Datenverarbeitung, des Internets und sozialer Netzwerke bergen Chancen, aber auch Gefahren für den Einzelnen und für die Gesellschaft. Der Kriminalprävention und dem Verbraucherschutz sind dadurch neue Herausforderungen erwachsen. Neue Informations- und Kommunikationstechniken verändern nicht nur Arbeitsweisen auf vielen Gebieten, auch im Bereich der Kunst und Kultur, sondern auch das Verhältnis und Zusammenleben der Menschen untereinander. Hieraus ergeben sich Fragen, deren Stellung und Beantwortung Aufgabe dieses Vereins ist.
Der Verein fördert und unterstützt Vorhaben der Bildung und Volksbildung zu diesen neuen technischen und sozialen Entwicklungen oder führt diese durch.
Der Vereinszweck soll unter anderem durch folgende Mittel erreicht werden:
\begin{enumerate}
\item Regelmäßige öffentliche Treffen und Informationsveranstaltungen,
\item  Veranstaltungen und/oder Förderung von virtuellen sowie regionalen und internationalen Treffen,
\item  Öffentlichkeitsarbeit in allen Medien,
\item  Informationsaustausch mit den in der Datenschutzgesetzgebung vorgesehe- nen Kontrollorganen,
\item  Förderung des schöpferisch-kritischen Umgangs mit Technologie,
\item  Erfahrungsaustausch unter den Mitgliedern und mit Dritten.\
\end{enumerate}
\item Der Verein verfolgt ausschließlich und unmittelbar gemeinnützige Zwecke 
im Sinne des Abschnitts ``Steuerbegünstigte Zwecke'' der Abgabenordnung. 
Er darf keine Gewinne erzielen; er ist selbstlos tätig und verfolgt 
nicht in erster Linie eigenwirtschaftliche Zwecke. Die Mittel des Vereins werden ausschließlich und unmittelbar zu den satzungsgemäßen 
Zwecken verwendet. Die Mitglieder erhalten keine Zuwendung aus den 
Mitteln des Vereins. Niemand darf durch Ausgaben, die dem Zwecke des 
Vereins fremd sind oder durch unverhältnismäßig hohe Vergütungen begünstigt werden.
\end{enumerate}
\section{Mitgliedschaft} \label{sec:mitgliedschaft}
\begin{enumerate}
 \item Vereinsmitglieder können natürliche Personen ab 18 Jahren
 und  juristische Personen werden. Natürliche Personen, die in
 ihrer Geschäftsfähigkeit beschränkt sind, können nur mit
 schriftlicher Zustimmung eines gesetzlichen Vertreters
 Mitglied werden. Es wird unterschieden in Vollmitglieder, in
 Folge nur Mitglieder genannt und Fördermitglieder.
 \item Die Mitgliedschaft wird durch schriftliche
 Beitrittserklärung gegenüber einem Vorstandsmitglied
 begründet.
 \item Die Mitgliedschafft wird nach dem einreichen im nächsten zeitlich liegenden Plenum durch die einfache Mehrheit beschlossen und ist von den Vorstandsmitgliedern zu bestätigen.
 \item Veränderungen im Mitgliederbestand werden vom Vorstand
 den Mitgliedern bekanntgemacht.
 \item Die Rechte von Fördermitgliedern werden durch die
 Mitgliederversammlung festgelegt.
\end{enumerate}
\section{Beendigung der Mitgliedschaft} \label{sec:beendigung_der_mitgliedschaft}
\begin{enumerate}
 \item Die Mitgliedschaft endet mit dem Tod des Mitgliedes, durch
 freiwilligen Austritt, Ausschluss aus dem Verein oder Verlust
 der Rechtsfähigkeit der juristischen Person.
 \item Der freiwillige Austritt ist jederzeit zulässig und
 erfolgt durch schriftliche Erklärung gegenüber dem Verein. Er
 wird mit dem Zugang der Erklärung wirksam.
 \item Ein Mitglied kann, wenn es gegen die Vereinsinteressen in
 grober Weise verstoßen hat, durch Beschluss des Vorstandes aus
 dem Verein ausgeschlossen werden. Als grober Verstoß gegen
 die Vereinsinteressen gilt auch die übermäßige
 Inanspruchnahme der Vereinsmittel durch einzelne Mitglieder,
 wenn dadurch der Vereinszweck insgesamt gefährdet wird. Dem
 Mitglied sind der beabsichtigte Ausschluss und die Gründe
 dafür rechtzeitig durch ein Vorstandsmitglied mitzuteilen;
 ihm ist mit einer Frist von mindestens 14 Tagen vor der
 Beschlussfassung Gelegenheit zu geben, sich gegenüber dem
 Vorstand zu äußern. Fasst der Vorstand innerhalb eines Monats
 seit der ersten Mitteilung keinen Beschluss, verfällt die
 Wirkung der ersten Mitteilung. Der Beschluss ist dem
 betroffenen Mitglied durch ein Vorstandsmitglied 
 mitzuteilen; im Falle des Ausschlusses sind ihm auch die
 Gründe mitzuteilen.
 \item Gegen den Ausschließungsbeschluss des Vorstandes steht dem
 Mitglied das Recht der Berufung an die Mitgliederversammlung
 zu. Die Berufung muß innerhalb eines Monat ab Zugang des
 Ausschließungsbeschlusses beim Vorstand schriftlich oder
 mündlich zur Niederschrift eingelegt werden. Sofern der
 Ausschließungsbeschluss einstimmig gefasst wurde, ist es
 ausreichend, die Berufung der nächsten ordentlichen
 Mitgliederversammlung vorzulegen; andernfalls hat der
 Vorstand innerhalb von zwei Monaten die Mitgliederversammlung
 zur Entscheidung darüber einzuberufen. Geschieht dies nicht
 so gilt der Ausschließungsbeschluss als nicht gefasst. Wird die
 Berufung nicht fristgerecht eingelegt, gilt die
 Mitgliedschaft ab dem Zeitpunkt der Beschlussfassung über den
 Ausschluss als beendet.
 \item Die Rechte und Pflichten des betroffenen Mitgliedes ruhen
 beitragsfrei vom Zeitpunkt der ersten Mitteilung (Absatz 3)
 bis zur endgültigen Entscheidung über den Ausschluss.
 \item Die Fördermitgliedschaft endet ebenso durch die in §5
 Absatz 1-3 genannten Gründe, außerdem durch Ausbleiben der
 Mitgliedsbeiträge. Ein Recht auf Berufung ist ausgeschlossen.
\end{enumerate}
\section{Mitgliedsbeiträge} \label{sec:mitgliedsbeitraege}
\begin{enumerate}
 \item Der Verein erhebt von seinen Mitgliedern und Fördermitgliedern einen Mitgliedsbeitrag. Näheres regelt die Beitragsordnung.
\end{enumerate}
\section{Organe des Vereins} \label{sec:organe_des_vereins}
\begin{enumerate}
 \item Vereinsorgane sind der Vorstand, das Plenum und die Mitgliederversammlung.
\end{enumerate}
\section{Vorstand} \label{sec:vostand}
\begin{enumerate}
 \item Der Vorstand im Sinne des §26 BGB besteht aus dem
 Vorsitzenden, einem Stellvertreter des Vorsitzenden und dem
 Zahlmeister.
 \item Der Verein wird gerichtlich und außergerichtlich von je
 zwei Vorstandsmitgliedern gemeinsam vertreten.
 \item Der Vorstand kann voll geschäftsfähige Vereinsmitglieder
 schriftlich bevollmächtigen, den Verein zu vertreten. Die
 Vollmachtsurkunde muß den Vertretungsberechtigten und den
 Umfang der Vertretungsmacht genau bezeichnen; sie ist von
 allen Vorstandsmitgliedern eigenhändig zu unterzeichnen. Der
 Inhalt der Vollmacht ist den Mitgliedern bekanntzumachen.
 \item Der Vorstand tagt mindestens einmal im Quartal öffentlich.
 Der Vorstand ist mit zwei Mitgliedern beschlussfähig.
\end{enumerate}
\section{Aufgaben und Zuständigkeit des Vorstandes}
\begin{enumerate}
 \item Dem Vorstand obliegt die Durchführung der Beschlüsse der
 Mitgliederversammlung und die Geschäftsführung des Vereins.
 \item Der Vorstand entscheidet zwischen den
 Mitgliederversammlungen über alle Angelegenheiten des Vereins
 durch Beschluss; die Beschlüsse des Vorstandes sind für alle
 Mitglieder verbindlich, sofern und solange sie nicht von der
 Mitgliederversammlung aufgehoben werden.
\item Der Vorstand hat die Mitglieder über seine Beschlüsse nach 
 §16 Absastz 3 zu informieren.
\end{enumerate}
\section{Wahl des Vorstandes} \label{sec:wahl_des_vorstandes}
\begin{enumerate}
 \item Der Vorstand wird von der Mitgliederversammlung gewählt.
 Alle Vorstandsämter werden direkt bestätigt oder neu gewählt.
 Vorstandsmitglieder können nur voll geschäftsfähige
 Mitglieder des Vereins werden.
 \item Der Vorstand bleibt bis zu seinem Rücktritt oder einer
 Neuwahl im Amt. Mit Beendigung der Mitgliedschaft im Verein
 endet auch das Amt des Vorstandes.
 \item Endet das Amt eines Vorstandsmitglieds anders als durch
 Neuwahl, muß der Vorstand innerhalb von zwei Monaten die
 Mitgliederversammlung einberufen. Bis zur Wahl eines neuen
 Vorstandsmitglieds können die verbleibenden
 Vorstandsmitglieder das frei gewordene Amt durch einstimmigen
 Beschluss wahlweise einem anderen Vorstandsmitglied
 zusätzlich zuweisen oder ein Vereinsmitglied als Ersatz in
 den Vorstand berufen.
\end{enumerate}
\section{Vorstandssitzungen} \label{sec:vorstandssitzungen}
\begin{enumerate}
 \item Der Vorstand beschließt in Sitzungen, die von jedem
 Vorstandsmitglied einberufen werden können. Die Vorlage einer
 Tagesordnung ist nicht notwendig.
 \item Der Vorstand ist beschlussfähig, wenn mindestens zwei
 seiner Mitglieder anwesend sind. Der Vorstand entscheidet mit
 Stimmenmehrheit. Bei Stimmengleichheit entscheidet die Stimme
 des Vorsitzenden; sofern dieser abwesend ist oder sich der
 Stimme enthält, ist bei Stimmengleichheit der Beschluss nicht
 zustande gekommen.
 \item Der Vorstand gibt sich eine Geschäftsordnung.
 \item Beschlüsse des Vorstandes werden gemäß §15 Absatz 3
 bekannt gemacht.
\end{enumerate}
\section{Mitgliederversammlung} \label{sec:mitgliederversammlung}
\begin{enumerate}
 \item In der Mitgliederversammlung hat jedes Mitglied eine Stimme; 
 Mitglieder, die mehr als drei fällige Monatsbeiträge nicht gezahlt 
 haben, haben kein Stimmrecht. Für juristische Personen übt das 
 Stimmrecht deren gesetzlicher Vertreter  oder eine mit schriftlicher 
 Vollmacht versehene voll geschäftsfähige Person aus. Fördermitglieder
 haben kein Stimmrecht.
 \item Die Übertragung der Ausübung des Stimmrechts auf andere
 Mitglieder ist unzulässig.
 \item Die Mitgliederversammlung ist zuständig für Entlastung
 und Wahl des Vorstandes, Änderung der Satzung, Ernennung von
 Ehrenmitgliedern, sowie für alle weiteren Angelegenheiten des
 Vereins, die sie zum Gegenstand ihrer Beratung und
 Beschlussfassung macht. Ihre Beschlüsse sind für alle
 Vereinsmitglieder und den Vorstand verbindlich.
 \item Mindestens einmal im Jahr - möglichst im ersten Quartal -
 soll eine ordentliche Mitgliederversammlung stattfinden. Sie
 wird vom Vorstand mit einer Frist von zwei Wochen unter
 Angabe der Tagesordnung durch schriftliche Einladung gemäß
 §15 Absatz (1) einberufen; das Einladungsschreiben wird zudem
 den Mitgliedern bekannt gemacht. Fördermitglieder werden nicht
 per Einladungsschreiben eingeladen, erhalten aber eine diesbezügliche
 E-Mail. Die Tagesordnung ist zu ergänzen,
 wenn dies ein Mitglied bis spätestens eine Woche vor dem
 angesetzten Termin schriftlich fordert. Die Ergänzung ist zu
 Beginn der Versammlung bekannt zugeben. Eine außerordentliche
 Mitgliederversammlung ist einzuberufen, wenn dies von
 derjenigen Zahl von Mitgliedern, die für die
 Beschlussfähigkeit gemäß Abs. 5 zum Zeitpunkt des Verlangens
 erforderlich wäre, verlangt wird; das Verlangen ist
 schriftlich, mit der erforderlichen Anzahl von Unterschriften
 versehen, an den Vorstand zu richten.
 \item Die Mitgliederversammlung ist beschlussfähig, wenn sie
 ordnungsgemäß einberufen wurde und mindestens ein Fünftel der
 stimmberechtigten Mitglieder anwesend ist. Ist weniger als
 ein Fünftel der stimmberechtigten Mitglieder anwesend, kann
 die  Mitgliederversammlung erneut einberufen werden; sie ist
 dann ohne Rücksicht auf die Zahl der anwesenden Mitglieder
 beschlussfähig. Auf diesen Umstand wird in der Einladung
 hingewiesen.
 \item Beschlüsse der Mitgliederversammlung werden mit einfacher
 Mehrheit der abgegebenen gültigen Stimmen gefaßt,
 Stimmenthaltungen bleiben außer Betracht. Satzungsänderungen
 bedürfen einer 3/4 Mehrheit der anwesenden, stimmberechtigten
 Mitglieder. Hierbei kommt es auf die abgegebenen gültigen
 Stimmen an. Für die Änderung des Vereinszwecks ist die
 Zustimmung aller stimmberechtigten Mitglieder erforderlich.
\end{enumerate}
\section{Protokollierung} \label{sec:protokollierung}
\begin{enumerate}
 \item Über den Verlauf von Sitzungen und entscheidungsfindenden
 Prozessen ist ein Protokoll anzufertigen, das von dem
 Versammlungsleiter und dem Schriftführer (Protokollführer) zu
 unterzeichnen ist.
\end{enumerate}
\section{Offenlegung der Bücher} \label{sec:offenlegung_der_buecher}
\begin{enumerate}
 \item Auf schriftlichen Antrag von mindestens drei
 Vereinsmitgliedern sind die Kassenbücher binnen eines Monats
 seit dem Antrag den Antragstellern offen zu legen und zu
 erläutern.
\end{enumerate}
\section{Auflösung des Vereins} \label{sec:aufloesung_des_vereins}
\begin{enumerate}
 \item Die Auflösung des Vereins ist durch Beschluss der
 Mitgliederversammlung mit 4/5 Mehrheit der stimmberechtigten
 Mitglieder herbeizuführen.
 \item Im Falle der Auflösung des Vereins oder bei Wegfall
 steuerbegünstigter Zwecke fällt das Vermögen des Vereins an
 den cbase e.V., zwecks Verwendung für Förderung und
 Weiterbildung von elektronischer Kommunikation, Kunst und
 Kultur.
 \item Vor Durchführung der Auflösung und Weitergabe des noch
 vorhandenen Vereinsvermögens ist zunächst das Finanzamt zu
 hören.
\end{enumerate}
\section{Mitteilungen und Bekanntmachungen des Vereins}
\label{sec:mitteilungen_und_bekanntmachungen_des_vereins}
\begin{enumerate}
 \item Gegenüber Vereinsmitgliedern gelten schriftliche und elektronische
 Mitteilungen und Einladungen dann als erfolgt, wenn sie an die letzte
 mitgeteilte (E-Mail-) Adresse versendet wurden, oder wenn die Mitteilung
 oder Einladung gemäß Absatz (\ref{abs:bekanntmachungen}) bekanntgemacht wurde.
 \item Andere Mitteilungen gelten als erfolgt, wenn sie dem
 Mitglied zur Kenntnis gelangt sind, oder wenn die Mitteilung
 gemäß Absatz (\ref{abs:bekanntmachungen}) bekanntgemacht wurde.
 \item \label{abs:bekanntmachungen}
 Bekanntmachungen an die Mitglieder werden für mindestens
 zwei Wochen zur Kenntnisnahme in den Vereinsräumen
 angeschlagen; darüber hinaus sollen Bekanntmachungen an die 
 Vereinsmitglieder auch auf andere geeignete Weise verbreitet
 werden.
\end{enumerate}
\section{Plenum}
\label{sec:plenum}
\begin{enumerate}
 \item Das Plenum bestehend aus den Mitgliedern des Vereins trifft sich regelmäßig mindestens ein mal im Quartal.
 \item Der Vorstand gibt den Termin des Plenums bekannt.
 \item Das Plenum ist mit der einfachen Mehrheit beschlussfähig.
 \item Es kann keine Änderungen in der Satzung bestimmen, oder Beschlüsse der vorangegangen Mitgliederversammlung ändern oder aufheben.
\end{enumerate}
\end{document}
